\documentclass[a4paper]{article}

\usepackage[T1]{fontenc}
\usepackage[swedish]{babel}
\usepackage[utf8]{inputenc}
\usepackage{ae}
\usepackage{graphicx}
\usepackage{hyperref}
\usepackage{multirow}
\usepackage[table]{xcolor}


\usepackage{fancyvrb}
\fvset{tabsize=4}
\fvset{fontsize=\small}

\title{Projektplan}

\author{DAT290 Group 14}
\date{\today}

\begin{document}

\maketitle
\hrule
\ \\[0.2cm]
\begin{center}
\begin{minipage}{0.5\textwidth}
		\textbf{Gruppmedlemmar:} \\
		Sebastian Danckwardt \\
		Isac Holm \\
		Zaid Haj Ibrahim \\
		Alamin Alreda \\
		Edvin Svahn
\end{minipage}
\end{center}
\clearpage

\tableofcontents
\newpage

\section{Syfte}
Projektet syftar till att skapa ett lättanvändligt och billigt alternativ för larmsystem för att skydda mot stöld. Detta då endast 29\% av sveriges befolkning idag använder ett larmsystem[1]. Larmsystem tenderar att vara komplicerade och dyra, genom att skapa ett system med en billig md407 dator kan man sänka kostnaderna för denna produkt och bekämpa brott i alla områden i Sverige. Bara under 2021 har 72 884 fall anmälts för inbrottsstölder och saknad av larmsystem är den största indikatorn på om ett hus är i risk för inbrott.

% Varje månad säljs 2500 inbrottslarm för hem och företag. Enligt polisen är larm till radhus ett av de bästa sätten att skydda sig mot inbrott, för att tjuvar väljer hus utan larm. 

%tekniska målen
%varför projektet ska genomföras ?
%statistics:
%Homes without a security system are 300% more likely to be broken into and burglarized. (Alarms.org)

%83% of would-be burglars check for the presence of an alarm system before attempting a break-in. (FBI)

%Police solve only 13% of reported burglary cases. (Pew Research Center)


\section{Mål}
Målet med projektet är att skapa ett larm/lås-system med två olika larmenheter: den ena ska kunna upptäcka ifall en dörr står öppen och den andra ska detektera rörelse och vibrationer. Larmsystemet ska konstrueras med en centralenhet så att flera larmenheter kommunicerar med en central styrenhet. Detta ska möjliggöra centraliserad kontroll och kalibrering av flera periferienheter i ett större larmsystem. Yttligare en enhet ska simulera en defekt larmenhet i syfte för testning.
%En centralenhet ska ingå i larmsystemet så att flera larmenheter kan kommunicera.
\\\\
Dörrenheten ska först larma lokalt med en röd lysdiod och efter en bestämd tid ska enheten larma till centralenheten. För att stänga av larmet behöver en fyrsiffrig kod matas in på en keypad.

%vad ska konstrueras
\section{Bakgrund}
Ett larmsystem är utformat för att upptäcka inbrott i en byggnad eller ett område. Säkerhetslarm används i bostäder, industriella och militära fastigheter för skydd mot inbrott och stöld av egendom. Vissa larmsystem är enkelbyggda och har bara en uppgift att göra medan andra larmsystem är mer komplicerade och ansvarar för flera uppgifter samtidgit, t ex brand och inbrott, men oavsett hur ett larmsystem är byggt så brukar de ha en centralenhet. Centralenhetens huvudskalig uppgift är att varna "ägaren" och ta olika åtärd vid olika fall.
\\\\
I det här projektet kommer vi att konstruera ett larmsystem som använder tre 32 bitars ARM-processorer (MD407). Systemet kommer att bestå av en centralenhet och två periferienheter, där varje enhet använder sig av en MD407-chip. Första enheten kommer att ansvara för dörrarnas larm medan den andra enheten kommer att ansvara för rörelselarm och vibrationslarm. Centralenheten kommer att hantera all kommunikation med användaren och samordning mellan periferienheterna. 


\subsection{Referenser}
\renewcommand*{\refname}{}
\vspace{-1cm}

\begin{thebibliography}{99}
\bibitem{engproc1} Through the eyes of a burglar: Study provides insights on habits and motivations, importance of security. \emph{ScienceDaily}. University of North Carolina at Charlotte. (2013, Maj 16). www.sciencedaily.com/releases/2013/05/130516160916.html (accessed September 7, 2022)

\bibitem{engproc2} Fungerar larm mot inbrott. \emph{Stöldskyddsföreningen}. (2019, Oktober 8). https://www.stoldskyddsforeningen.se/fungerar-larm-mot-inbrott/#:~:text=Enligt%20den%20demoskopunders%C3%B6kning%20som%20SSF,fungerande%20Grannsamverkan%20i%20sitt%20omr%C3%A5de

\bibitem{engproc3} Modernaförsäkringar.(2011 Maj 2). modernaforsakringar.se https://www.modernaforsakringar.se/om-moderna/pressmeddelande-och-nyheter/fa-larmar-hemma-trots-stor-oro-for-inbrott/ (accessed September 7, 2022)
\end{thebibliography}


\subsection{Tekniska förutsättningar}

Syftet med detta projekt är att simulera larmsystemsfunktionalitet, därför kommer all fokus och utveckling ske på mjukvaran. Mjukvaran kommer att skrivas med C-språket och all testning av koden och dess funktionaliter kommer att ske på ETERM8. De huvudsakliga punktern som behöver nyutvecklas för mjukavaran är: kommunikationen mellan periferienheterna och centralenheten, kommunikationen mellan centralenheten och användaren, dataläsning av sensorna och 


\textbf{Hårdvara:} 
\\
3x MD407 kort \\
1x Avståndsmätare (ultraljud), HC-SR04 \\
1x Vibrationssensor, "Flying-Fish" SW-18010P \\
1x Keypad \\
1x 7-segmentsdisplay \\
2x 4-polig RJ-11 kabel (används för CAN-bussen) \\
1x RJ-11 förgrening \\
2x Tiopolig flatkabel \\
3x USB-kabel \\
1x Kopplingsplatta \\
kopplingskablar\\
diodlampor\\
\\\\
\section{Systemöversikt}
Systemet är uppbyggt av en centralenhet och två periferienheter och varje enhet är kopplad till olika instrument beroende på dess funktionalitet (figure 1). Periferienhet 1 är kopplad till en kopplingsplatta, på plattan finns det lampor och dörr-strömbrytare kopplade. Lamporna kommer lysa rött efter att dörren har varit öppen en viss tid eller grönt om ingen larm gått. Dörr-strömbrytarna kommer att simulera dörröppning eller dörrstängning. \\
\\\\
Periferienhet 2 är kopplad till en avståndsmätare (HC-SR04) och en vibrationssensor (SW-18010P). Avståndmätaren kommer att skicka en signal om avståndet har ändrats. På samma sätt kommer vibrationssensor skicka en signal om den har känt av några vibrationer. \\
\\\\
Centralenheten är kopplad till en dator, en 7-segementsdispaly, en keypad och både periferienheterna. Kommunikationen mellan periferienheterna och centralenheten kommer att ske över en CAN-buss (Controller Area Netwokr

\begin{figure}[b]
\includegraphics[scale=0.3]{projekt (6).jpg}
%\caption {}
\label{fig:drawing}
\end{figure}

\\\\
\section{Resursplan}
\begin{itemize}
 \item Sebastian Dankckwardt (Gruppledare, Kodstandardansvarig, teknisk dokumentansvarig)
 \\swesebbe3336@gmail.com
 \item Isac Holm (Resursansvarig)
 \\holmisen88@gmail.com
 \item Edvin Svahn (Administrativ dokumentansvarig)
 \\edvinsva@chalmers.se
 \item Zaid Hajibrahim (Planeringsansvarig)
 \\zaid3019@hotmail.com
 \item Alamin Alreda (Testansvarig)
 \\amin.reda@hotmail.com
\end{itemize}

Kommunikation utanför mötestid kommer främst att ske via en discordchatt men alla medlemmar kan även nås via telefon eller epost. All hårdvara är tillgänglig vid behov i sal ED4217 inlåst i ett kassaskåp där alla gruppmedlemmar har tillgång till koden. Mjukvaran som finns till hands är GitHub, Codelite, eterm8 och STM-biblioteket. Lokalerna som kan disponeras är alla lokaler som kan bokas på Chalmers.
%mejla kanske mentor om ovanstående stämmer.
\\\\
I det fallet att en gruppmedlem vill jobba på distans kan medlemmen vara medverkande på möten genom discords röstchatt. Medlemmen kan även bidra till skriftliga dokument på distans i ett delat LaTeX dokument via overleaf. För att jobba på programmeringen kan medlemmen pusha sin skrivna kod från CodeLite till det gemensamma GitHub repot.


\section{Aktiviteter} % Work breakdown

\begin{description}
 \item[Projektplanering] Alla medlemmarna ska var medvetna om projektsmål samt vad som förväntas av dem. Samtliga meddlemmar ska godkänna hur projektetsdelmomenterna skall fördelas samt utföras under projektetsundergång. Antal timmar som skall läggas på projektet är 1000 timmar totallt. Projektmöten kommer hållas två gånger i åtta veckor. med fem medlemmar 150 timmar.


\begin{tabular}{ |p{0.5cm}||p{4cm}|p{2cm}|p{4cm}|  }
\hline
  \rowcolor{lightgray}
  Nr & Moment & Tidåtsgång & Ansvar\\
 \hline
 1  &  Projektmöten             & 150  &planeringsansvarig\\
    \rowcolor{lightgray}
 2  &  Projektledning           & 50   &Gruppledare\\
 3  &  Projektsrapport          & 200  &Samtliga medlemmar\\
    \rowcolor{lightgray}
 4  &  Dokumentgranskning       & 20   &Dokumentsansvarig\\
 5  &  Kodstruktur förstärkare  & 50   &Kodansvarig\\
    \rowcolor{lightgray}
 6  &  demonstration            & 20   &Samtliga medlemmar \\
 7  &  Oppositionskommentar     & 30   &Samtliga medlemmar\\
    \rowcolor{lightgray}
 8 &  Programmering             & 500  &Samtliga medlemmar\\
 9 &  Konstruktions förstärkare & 100  &Resursansvarig\\
 \hline
\end{tabular}




 En planering utav arbetet som skall definiera uppgifter samt milstolpar i projektet.

\begin{itemize}
 \item Projektmöten hålls i genomsnitt  gånger varje vecka. Mötet kan varierar mellan två till fyra timmar lång. Ämne som diksuteras i möten är bland annat gruppdynamik, processen, upplysningar och feedback från mentorn.
 
 \item Projektledning styrs av gruppledaren övervaka alla uppgifterna som rör gruppdynamiken. 
 
 \item Dokumentgranskning utförs av tekniskdokumentansvarig sköter språk, stil och struktur i projektrapporten. 
 
 \item kodstruktur förstärkare sköts av kodansvarig rättar till koden så att den är buggfri. 
 
 \item Oppositionskommentar görs av samtliga medlemmar i sista perioden av projekten 
 
 \item Programmering har det störsa delen av tidåtsgång
 \item Konstruktions förstärkare

\end{itemize}

 \item[Kravspecificering] Krav diskuteras, eliciteras och omformuleras.

 \item[Högnivå-design] Design av systemet för en översiktlig struktur, som resulterar i lågnivå-design senare.

 \item[Testplanering] En testplan beskriver hur programmet kommer fungera i olika testscenarion samt förhållanden.
 
 \item[Implementering] Utgående från designen sker implementeringen i programkod. Olika delar av systemet implementeras parallellt.

 \item[Kvalitetsförsäkring] När programmet är färdigskrivet måste programmet testas för att försäkra kunden att programmet uppfyller kraven ställda på kravspecifikationen.
 
 \item[Rapportering] Rapporteringen sker internt inom gruppen fär att färsäkra gruppmedlemmarna om respektive person är färdiga med sina uppgifter eller ej.

 \item[Projektplan] Övergripande planen för hur projektet ska genomföras. Detta dokument. Dokumentansvarig: Johan Lundström.

 \item[Kravspecifikation] Kundens lista på specifika krav, vilket produkten begärs kunna lösa. Dokumentansvarig: Mazdak Farzone.

 \item[Design] Programdesign utifrån kravspecifikationen.
 
 \item[Programkod] Systemimplementation i Java som formar mjukvaran.

 \item[Testplan] Dokumentet som beskriver interna samt extarna tester som utförs på programmet. Dokumentansvarig: Björn Lennernäs.

 \item[Testprotokoll] Ifyllda protokoll över de tester man utfört på programvaran.
\end{description}
\
\
\section{Milstolpar}
\begin{center}
     \begin{tabular}{|c|c|c|}
      \hline
      Vecka & Beskrivning & Datum \\
      \hline
      1 & Projektplan Inlämnad & 2022-09-09 \\
      \rowcolor{lightgray}
      2 & RapportUtkast 1 Inlämnad & 2022-09-30 \\
      n/a & Huvudenhet kommunicera med alla periferienheter & n/a \\
      \rowcolor{lightgray}
      n/a & Få CAN-bussen att fungera & n/a \\
      n/a & rörelsedetektor kan ge signal & n/a \\
      \rowcolor{lightgray}
      n/a & vibrationssensor kan ge signal & n/a \\
      3 & Oppositionsrapport Inlämnad & 2022-10-06 \\
      \rowcolor{lightgray}
      4 & RapportUtkast 2 Inlämnad & 2022-10-19 \\
      5 & Demo & Dunno \\
      \rowcolor{lightgray}
      6 & Slutrapport Inlämnad & 2022-10-30 \\
      \hline
 
     \end{tabular}
 \end{center}
\
\
\section{Kommunikationsplan}
För att alla i gruppen ska hållas informerade så behöver vi kontinuerligt uppdatera varandra om projektets gång, deadlines, planeringar, mm. För detta krävs det att vi har ett enkelt sätt att kommunicera på. Efter åtanke betämmdes Discord som primära kommunikationsplatform på grund av dens många funktioner som är användbara för detta projekt fildelning, flera te

Det äroch kan lätt hittast.  efterå också viktigt att informationen är lätt att hitta, uppdatera, samt notifierar gruppmedlemmerna så att ingen missar ny information. 
Kommunikationsplatformen Discord blev vald på grund av dens användarvänlighet och 

Gruppen kommer att kommunicera genom discord-applikationen.

Anledning att discord
blev vald är den erbjudar sina kunder service utan kostand, lätt att skapa olik server

Vi skapar flera s.k text kanaler där medlemmarna kan kommunicera och informerar varann gällande varje
fråga (kanal). Detta kan medför till en enklare och informativt kommunikationsmedia. Discord
offrar möjlighet till voice-chat eller även screen-sharing chat, vilket kan ersätta fysiska möten 
i fall av sjukdom eller liknande. Med det talat, fysiska möte kommer även att förekomma i tillfälle medlemmarna kommer överens om. 



\section{Kvalitetsplan}
Gruppen ser till att mötas regelbundet. På mötena granskas arbete utfört av gruppmedlemmar, speciellt inför varje extern granskning. På mötena utses även ansvariga för olika delar av projektet, samt vilka som skall arbeta på vilket dokument eller program. Ansvaret ligger sedan hos den ansvariga för dokumentet/programmet att se till att de arbetare personen har tillgång till utför ett effektivt arbete, och att arbetet är slutfört inför nästa deadline.

För att se till att kvalite på projektet hålls ska även tester hållas. Dessa ska dokumenteras och utföras fullt för att förhindra att delar som är trasiga läggs till på ställen som senare kommer kunna skapa problem. Testerna ska utföras med hjälp av en testmall, denna ska fyllas ut när ett test genomförs så att de med säkerhet görs noggrant och utförligt. Dessa tester ska göras kontinuerligt under utvecklingsperioden.
\begin{description}
\item[Komponent:] 
Dokumentera exakt vilken komponent som ska testas.

\item[Testsyfte:] Vad är syftet med testet?

\item[Förväntade resultat:] Vad är det förväntade resultatet?

\item[Utförande:] Hur ska testet utföras? Om det är kod som ska testas i en IDE, vilka tester är skrivna? Om det är hårdvara som ska testas, vilka extra verktyg behövs och vad ska de användas till?

\item[Resultat:] Beskriv resultatet utan att ge någon värdering.

\item[Analys:] Stämde resultat överens med det förväntade resultatet? Om ja, täckte testet allt som behövdes testas? Behövs det fler tester? Om inte, vad skiljer sig? Behöver det fixas?
\end{description}
Förutom detta ska även koden granskas av kodstandardansvarige så att den håller sig till konventioner och standarder.


% BS som ska omskrivas sen
Tidsplan:

Det preliminära planet är att lägga i genomsnitt 22 timmar per vecka. detta 
är inklusive föreläsningarna, möten som handlar om projektarbete och övningslektioner.
Tiden kan varierar beroende på statusen på projekten undergågn, tycker en medlem att
det finns behov att lägga mer tid, då måste intyga resten av medlemmarna om tid/plats som passar majoriteten.

Mötesplan:

Varje vecka hålls ett formellt möte med handledaren. möteskallelse är gruppledarens
ansvar att skicka innan varje möte. I mötet diskuteras gruppo samt individuella planerignen
om olika delmomententerna. Tid och plats på möten är anmälda i gruppens huvudkommunukationserver.

Dokument är skapat och alla medlemmar ska delta för att ställa upp tiderna och sätta upp ett schema.


\end{document}