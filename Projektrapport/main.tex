\documentclass{article}
\usepackage[utf8]{inputenc}
\usepackage[swedish]{babel}
\usepackage{authblk}
\usepackage[sorting=none]{biblatex}
\addbibresource{sources.bib}
\usepackage{graphicx}
\usepackage[T1]{fontenc}
\usepackage{ae}
\usepackage{multirow}
\usepackage[table]{xcolor}
\usepackage{pgfgantt}
\usepackage{pdfpages}

\title{\textbf{Larmsystem med MD407}\\ 
\hspace{10cm}
\hrule
\hspace{10cm}
Implementation av rörelsesensor, avståndsmätare och simulerad dörrenhet på MD407-mikrodator}
    
\author{\\Alamin Alreda, Epost: amin.reda@hotmail.com, Sektion: Datateknologi \\Sebastian Danckwardt, Epost: swesebbe3336@gmail.com, Sektion: Datateknologi\\Zaid Haj Ibrahim, Epost: zaidh@chalmers.se, Sektion: Datateknolog \\Isac Holm, Epost: isachol@chalmers.se, Sektion: Datateknologi\\Edvin Svahn, Epost: edvinsvahn01@gmail.com, Sektion: Datateknologi\\Handledare: Samuel Collier Ryder, Epost: samuel.collier.ryder@gmail.com}


\date{\today}
    
%LÄÖGGG TILL: handledare, e-post adresser, sektion, datum för senaste revision
    
\begin{document}

\maketitle
\newpage
\tableofcontents
\newpage
\section*{Ordlista}
\begin{description}

\item[ACK-meddelande] \emph{acknowledgement message} är ett meddelande som skickas i respons till att ett meddelande har tagits emot väl.

\item[Buss] En \emph{buss} är en kommunikationsport som tillåter överföring av data i form av elektriska signaler.

\item[CAN] \emph{Controller Area Network} är en databuss som tillåter kommunikation mellan mikrokontrollers. MD407 är försedd med två CAN-bussar.

\item[GPIO] \emph{General Purpose Input/Output} är en anslutningsport på MD407 där olika logiska enheter kan anslutas för att ta emot eller skicka data.

\item[HC-SR04] En ultraljudsbaserad avståndsmätare som används för att känna av om något står inom en viss  räckvidd. 

\item[Knappsats] En extern periferienhet med 16 knappar som används för inmatning av data till mikrokontroller.

\item[SW-18010] En fjäderbaserad vibrationssensor som används för att mäta vibrationer runtom sig.

\item[USART] \emph{Universal Synchronous/Asynchronous Reciever/Transmitter} är hårdvara som tillåter seriell kommunikation mellan mikrokontroller och PC.

\end{description}
 \newpage

\setcounter{page}{1}
\section{Introduktion}

Under året 2021 anmäldes 72 884 inbrottsstölder i Sverige vilket pekar på att det finns ett tydligt behov av larmsystem i dagens samhälle \cite{BRa}.
Trots den oro att ens egendom ska bli stulen som finns i allmänheten, så tillkännager endast 13 \% av unga mellan 23 och 35 år att de äger ett huslarm \cite{MoFor}.
En av de största anledningarna till detta är kostnaden, 20 \% av de som saknar ett hemlarm anger att priset på installation är för högt \cite{MoFor}.
Genom att skapa ett prisvärt larmsystem kan man erbjuda möjligheten till en tryggare tillvaro.

%NY INTRODUKTION ADDERAD

\subsection{Syfte}
Projektet syftar till att skapa ett kostnadseffektivt larmsystem som är lätt att använda.
Detta då endast 29 \% av Sveriges befolkning äger ett larmsystem \cite{SSF}.
Systemet är menat att installeras i bostäder, vilket kommer att sänka inbrott samt öka trygghet i framförallt låg-inkomsthushåll.
För att larmsystemet ska fungera säkert och korrekt, måste krav ställas på systemet.

\subsection{Mål}
Målet med projektet är att konstruera ett larm och låssystem med två periferienheter.
Den ena ska kunna upptäcka i fall en dörr står öppen, och den andra ska detektera om någonting rör sig inom en viss zon samt vibrationer i exempelvis ett fönster. 
Ett delmål är att ha en central styrenhet som ska utgöra den del av larmsystemet som har i uppgift att kommunicera med de perifera larmenheterna, den ska även kunna upptäcka ifall en periferienhet har kopplats bort från systemet.
Målet med detta är att möjliggöra centraliserad översikt och kalibrering av flera larmenheter i ett larmsystem.
Ytterligare ett delmål är att ha en enhet som ska agera defekt för att kunna testa ifall systemets kommunikation är robust nog för att hantera överflöd av data.\\
\\
Ett delmål med lägre prioritet är att centralenheten ska kunna konfigurera en ny periferienhet vid anslutning. När den centrala styrenheten startar ska en förfrågan skickas till de anslutna periferienheterna.
Därpå svarar varje periferienhet med dess typ och konfiguration för att sedan tilldelas ett unikt ID.
Dörrenhetens mål är att inledningsvis larma lokalt med en röd lysdiod när en larmad dörr öppnas. 
Dörrlarmet larmar till centralenheten efter att det lokala larmet har varit igång under en specifik kalibreringsbar tid vilket kan justeras av användaren. 
Exempelvis kan tidsperioden innan larmet aktiveras vara 4 sekunder för det lokala och 8 sekunder för det globala.\\
\\
Avståndsmätare ska larma centralt ifall något passerar framför den inom ett visst avståndsintervall, ett delmål med lägre prioritet är att även implementera ett lokalt larm för avståndsmätaren.
Vibrationssensorn ska detektera rörelse i form av vibrationer som antingen kan färdas i luften eller genom materialet som den är fäst vid.
Ett centralt mål är att det ska vara möjligt att låsa eller låsa upp dörren samt inaktivera dörrlarmet genom att mata in en fyrsiffrig kod på en knappsats ansluten till centralenheten.

\subsection{Arbetsmetod}
I arbetsmetod beskrivs gruppens arbetsprocess. Denna är indelad i 3 delar som har bidragit till den färdiga produkten. Förarbetet som gav en klarare bild av hur arbetet skulle utföras. Tekniskt genomförande ger en tydligare uppfattning av utvecklingsprocessen och avslutningsvis redogör testning och verifiering för hur testning tillämpades för att upptäcka buggar och svagheter i systemet.

\subsubsection{Förarbete}
Innan kod skrevs krävdes en plan för genomförandet av projektet.
Detta inkluderade fastställande av versionshanteringssystem, kommunikationsplatform, testning av produkten samt i vilken ordning larmsystemets olika delar skulle implementeras.
Versionshanteringssystemet som valdes var Github och som kommunikationsplatform valdes Discord.
För att testa systemet gjordes en testmall för hur ett test skulle genomföras. 
Ordningen som projektets olika delar skulle utföras i bestämdes med milstolpar. Först lades stort fokus på pereferienheterna. 
Endast när dessa närmade sin slutpunkt skulle fokuset skifta till CAN-protokollet för att kommunicera mellan dem. 
Motivationen till detta var att CAN-protokollet inte skulle behövas förrän pereferienheterna var klara, och att jobba på de enheterna skulle ge en klarare bild över hur CAN-protokollet skulle fungera som effektivast.

\subsubsection{Tekniskt genomförande}
När förberedelserna var klara delades projektet upp i tre delar som arbetades på parallellt. När en del var klar väntade man på att de andra delarna också skulle bli klara. 
Därefter gick gruppen igenom det som var kvar och delegerade arbetet.
\\
\\
Efter att dessa delar var färdiga gjordes CAN-protokollet samtidigt som buggar rättades i periferienheterna. Centralenheten tillverkades i samband med CAN-protokollet eftersom centralenheten krävde en fungerande kommunikationsmodell. Efter detta utvecklades störenheten för att simulera en defekt enhet.

\subsubsection{Testning och verifiering}
För att säkerställa kvalitén testades färdig kod kontinuerligt under arbetsprocessen med testmallar.
Rent praktiskt placerades ofta \emph{print-statements} på viktiga positioner i koden för att kontrollera att korrekt resultat producerades.
Då inget fel noterades var testet markerat som godkänt och ingen vidare åtgärd tillämpades. 
Däremot när resultat inte blev som testet förväntat söktes orsaken till felet upp och fixades omgående.
Utöver att testa kodens vanliga funktion har den testas mot olika typer av angrepp med hjälp en störenhet som skickar stora volymer data för att störa centralenhetens normala funktion. 
Störenheten användes också som en angreppsenhet, med hjälp av den så testades centralenhetens kapacitet att hantera återuppspelningsattacker samt imitationsattacker.

\section{Teknisk beskrivning}
I det här kapitlet redogörs en teknisk beskrivning av systemet. 
I den tekniska bakgrunden beskrivs vad samtliga komponenter har för syfte. Därefter ger Systemöversikt en övergripande förståelse för hur systemet är uppbyggt. 
Delsystem förklarar hur samtliga delar är uppbyggda.

\subsection{Teknisk bakgrund}
Larmsystem kan ha olika konstruktion beroende på säkerhetsbehov men samtliga system tillämpar flertal olika sensorer för att upptäcka inbrott. 
I fall en avvikelse upptäcks av en sensor ska larmet aktiverats och användare underrättas.
\\
\\
I konstruktionsarbetet är MD407-mikrodator den centrala komponenten. 
Detta är en laborationsdator med stabil hårdvara som har utvecklats i utbildningssyfte. 
Då datorn är utvecklad för utbildning finns det skydd mot elektrostatiska urladdningar, felkopplingar och mekaniska påfrestningar vilket gör den lämpad för utveckling av prototypsystem.
\\
\\
För att MD407-mikrodatorn ska kunna skicka och ta emot data från externa perifierienheter är den försedd med flera olika anslutningar som kan användas för detta. 
GPIO lämpar sig för koppling med enklare periferienheter och är den anslutningen som används för kommunikation med avståndsmätare samt vibrationssensor. 
Det finns 5 GPIO portar benämnda A-E. portarna har 16 kontaktstift vardera som kan konfigureras till att hantera indata eller utdata. 
Det finns även kontaktstift för ström samt jord som används av sensorerna.\\
\\
Då åtskilliga MD407-mikrodatorer behöver kommunicera med varandra är de försedda med två CAN-portar. Den sista anslutningen som används är USART, den används för att upprätthålla seriell kommunikation mellan en mikrodator och en PC.

\newpage
\subsection{Systemöversikt}

\\
\\
\noindent
Systemet i sin helhet består av tre delar, en centralenhet, en dörrenhet och en sensorenhet. 
Utöver dessa enheterna innehåller systemet övriga komponenter som bidrar till funktionalitet av systemet. 
Ett exempel på det är vara knappsatsen.
Ett av korten används som centralenhet vilken de övriga korten samt komponenterna kopplas till (figur 1). 
Kommunikation mellan samtliga MD407-mikrodatorer sker via en CAN-buss, medan kommunikation mellan centralenheten och PC sker över USART.
\\
\begin{figure}[h]
    \centering
    \includegraphics[scale=0.6]{Projektrapport/diagram (1).png}
     \caption {Blockschema av larmsystemet. Enkelpilarna visar en koppling över GPIO medans dubbelpilarna visar på koppling mellan dörrar.}
    \label{fig:drawing}
\end{figure}
\newpage
\\
\noindent
Dörrenheten ska vara ansluten till en kopplingsplatta och på plattan ska det finnas lysdioder.
Lysdioderna kommer lysa rött när dörrenheten larmar lokalt eller grönt om inget larm har gått.
\\
\\
Till sensorenheten kopplas avståndsmätare (HC-SR04) och vibrationssensor (SW-18010P).
Avståndsmätaren kommer att skicka en signal om det avmätta avståndet ligger inom ett visst tröskelvärde.
På samma sätt kommer vibrationssensorn skicka en signal om den har känt av några vibrationer över ett tröskelvärde.

\subsection{Delsystem}
Här beskrivs vad de olika delsystemen har för tekniska funktioner samt vilket syfte de uppfyller i systemet.

\begin{figure}[h]
    \centering
    \includegraphics[scale=0.05]{Projektrapport/central.png}
    \caption {Hårdvaran för centralenheten}
    \label{fig:drawing}
\end{figure}

\subsubsection{Centralenhet}
Centralenheten är systemets hjärna. 
Dess huvuduppgift är att upprätthålla en konstant kommunikation mellan sig själv och de anslutna enheterna. 
Centralenheten tar emot livstecken från anslutna enheter varje 500 millisekunder för att se till att enheterna fortfarande är anslutna. 
Ett larm utlöses om 1000 millisekunder passerar utan att centralenheten tar emot ett livstecken från en ansluten enhet.\\
\\
Centralenheten tar också hand om tilldelningen av ID till anslutna enheter. 
När en annan enhet startar skickar den ett "NEW ALIVE"-meddelande med sin enhetstyp till centralenheten. 
Centralenheten tar emot detta meddelande och svarar med ett ID för enheten, som den börjar använda för all framtida kommunikation.\\
\\
Utöver det sköter centralenheten all kommunikation mellan användaren och systemet via USART och en knappsats. 
När centralenheten startar uppmanar den användaren att ange ett lösenord. 
När det har angetts visar centralenheten sedan en meny med nio alternativ. 
Användaren kan då bland annat ändra lösenordet, uppdatera tröskelvärden och se status för alla enheter som är anslutna till centralenheten.

\subsubsection{Dörrenhet}
Varje dörr har två tröskelvärden som bestämmer hur länge kan dörren vara öppnad innan ett lokalt respektive centralt larm aktiveras. 
Totalt definieras åtta dörrar för dörrenheten, till varje dörr delas fyra GPIO-pinnar. Första pinnen är en strömbrytare som kopplas direkt till 3,3 volt spänning. 
Strömbrytaren kommer att anta värdet 1 om dörren är stängd respektive 0 om dörren är öppen. 
Andra pinnen kopplas till en grön lampa som lyser när dörren är stängd eller släcks när dörren öppnas. 
Tredje pinnen kopplas till en röd lampa för att indikera ett lokalt larm. 
Lampan ska lysa när dörren har varit öppen under en bestämd tid. 
Tiden bestäms genom centralenheten och ska släckas när dörren är stängd. 
Sista pinnen kopplas till ett lås för att antigen låsa eller låsa upp dörren, vilket också kan göras genom centralenheten. 
Alla lampor kopplas upp på en kopplingsplatta enligt figur 3.\\
\\
Om en dörr öppnas startas en timer som utlöser ett larm när en tid har passerat. 
Denna tid beror på de tröskelvärden som ställs in av centralenheten. 
Som standard startar det lokala larmet fyra sekunder efter att en dörr har öppnats och det globala larmet startar åtta sekunder efter att det lokala larmet har utlösts. 

\begin{figure}[h]
    \centering
    \includegraphics[scale=0.5]{Projektrapport/LED.png}
    \caption {Figuren visar hur en LED-lampa kopplas på en kopplingsplatta}
    \label{fig:drawing}
\end{figure}




\subsubsection{Sensorenhet}
Sensorenheten har två olika sensorer: en avståndsmätare och en vibrationssensor. 
Till avståndsmätaren delas två GPIO-pinnar. 
Första pinnen är trigg-pinnen och ansvarar för att skicka ultraljudspulser. 
Andra pinnen är eko-pinnen som antar en etta när ett eko kommer tillbaka och returnerar värdet noll om ingen läsning av eko sker. 
Skillnaden i tiden mellan när ultraljudspulsen skickades och när ekot har kommit tillbaka ges i mikrosekunder. 
Mätningen av avståndet sker enligt figur 4. 
Avståndsmätaren kopplas upp på en kopplingsplatta tillsammans med en röd lampa för att indikera att ett lokalt larm har inträffat. 
Trigg-pinnen sätts på i tio mikrosekunder innan den börjar vänta på eko-signaler. 
Hela mätningscykeln sker i ett intervall på 65 millisekunder. 
Sensorenheten kalibreras genom att en mätning utförs när det larmade området är tomt, för att få ett referensvärde. Referensvärdet används för att bestämma i vilka avståndsintervall som enheten ska larma centralt respektive lokalt. 
Det lokala larmet indikeras genom att tända den röda lampan och det centrala larmet indikeras hos centralenheten som får ett alarmmeddelande från sensorenheten. 
Referensvärdet, intervallet för lokalt larm och intervallet för centralt larm kan ändras genom centralenheten.
\begin{figure}[h]
    \centering
    \includegraphics[scale=0.5]{Projektrapport/ekvation.png}
    \caption {Ekvationen ger avståndet i centimeter }
    \label{fig:drawing}
\end{figure}
\\
Vibrationssensorn har fyra pinnar: en ström-pinne, en jord-pinne, en digital utdata-pinne och en analog utdata-pinne. 
Under detta projekt har det bestämts att bara använda den digitala utdata-pinnen eftersom det bara finns behov av att läsa om en vibration har inträffats eller inte.
Känsligheten för vibrationssensorn justeras med en komparator på själva sensorn. 
Sensorenheten skickar direkt ett larmmeddelande till centralenheten om en vibration har inträffat.

\begin{figure}[!tbp]
    \centering
    \includegraphics[scale=0.04]{Projektrapport/sensor.png}
    \caption {Sensorerna som är implementerade på sensorenheten. Hårdvaran till vänster är vibrationssensor och till höger är avståndsmätare}
    \label{fig:drawing}
\end{figure}


\subsubsection{Systemprotokoll}
CAN användes för att överföra meddelanden mellan enheterna i larmsystemet. 
Ett protokoll byggdes på toppen av CAN för att göra det mer pålitligt, säkrare och lättare att använda. 
CAN-meddelandet har två fält som används av det utökade protokollet: id-fältet och innehållsfältet. 
Id-fältet kan innehålla 11 bitar data som är uppdelade i fyra delar. 
Den mest signifikanta biten, bit 10, är inställd som en prioritetsbit. 
När den är satt till noll prioriterar protokollet ankomsten av det meddelandet. 
Bitar 9-6 är reserverade för meddelandetypen (se figur 6). 
Bitar 5-3 är för sändar-id och de tre minst signifikanta bitarna, 2-0, är för mottagar-id. 
Ett filter gjordes sedan baserat på detta id-fält och aktiverades i periferienheterna efter att de tilldelats sina id:n. 
Detta filter gör så att de endast accepterar meddelanden från centralenheten samt så måste meddelandenas mottagar-id vara deras id. 
Däremot tar centralenheten fortfarande emot meddelanden från alla enheter, detta för att förhindra att någon imiterar centralenheten. 
Det beror på att centralenheten granskar om ett meddelande skickas från någon med samma id och ifall det skulle uppstå en sådan situation utlöses ett larm. 
"Content"-fältet fylls i beroende på vilken typ av meddelande som skickats, vilket framgår av figur 6. 
Efter att ett meddelande har skickats sparas en kopia i en återutsändningsbuffert och en tidsstämpel sparas. 
Efter att en viss tid har gått utan att ett ack har mottagits skickas meddelandet igen. 
Det antas då att meddelandet inte nådde sin destination. 
Livstecken får dock inte någon bekräftelse eftersom de inte är livsviktiga och sänds upprepade gånger med korta intervaller. 
Livstecknets timeout uppnås också före ack timeout.

\begin{tabular}{|c|l|c|}
    \hline
    Bitmönster & Meddelandetyp & Innehåll \\
    \hline
    0000 & START ALARM & ... / DörrID \\
    0001 & STOP ALARM & ... /DörrID \\
    0010 & UNLOCK DOOR & DörrID \\
    0011 & NEW ALIVE & Enhetstyp \\
    0100 & NEW ALIVE RESPONSE & Enhetstyp hos mottagare \\ 
    &&och nytt ID \\
    0101 & RESET UNIT & ... \\
    0110 & SET DOOR ALARM TIME THRESHOLD & Nytt tröskelvärde \\
    0111 & SENSOR DISTANCE THRESHOLD & Nytt tröskelvärde \\
    1000 & RECALIBRATE & ... \\
    1001 & LIFESIGN & ... \\
    1010 & ACK & ... \\
    1011 & VIEW STATUS & ... \\
    \hline
\end{tabular}
\\
\begin{center}
\caption{Figur 6: Bitmönster för de olika meddelandetyper som används för CAN-kommunikation. Meddelandetyper med lägre nummer har högre prioritet. "..."\ betyder att innehållet är tomt.}
\end{center}

\subsubsection{Installering av larmsystem}
Installationen av systemet bildar en hierarki där centralenheten och dörr- och sensorenhet är basenheter. 
Basenheterna består av olika delsystem. 
Centralenheten är hjärnan för hela systemet.
Dess huvudsakliga uppgift är att leda ett kollektivt samarbete mellan alla enheterna. 
Detta innebär att larmsystemet inte skulle fullgöra sitt syfte utan centralenheten.\\
\\
Centralenheten kommunicerar med periferienheterna via CAN-protokollet som skickar över en RJ11 telefonkabel som i sin tur är kopplade till CAN-portarna på periferienheten. 
Därmed måste CAN-protokollet inkluderas för varje kommunicerande del i systemet. 
USB-koppling mellan PC och basenheterna är essentiellt för att upprätthålla USART-kommunikation och försörja enheterna med ström. 
Detta möjliggör dessutom att man kan ladda upp omarbetad mjukvara direkt till delenheterna. 
Delsystemet för timer genererar ett avbrott och används för tidtagning i övriga funktioner och för att se till att livstecken på kopplade periferienheter inte överskrider tidgränsen. 
Centralenheten använder sig av en knappsats. 
Knappsatsen är kopplad till centralenheten via GPIO portar och används för att manövrera menyn.


\begin{figure}[h]
    \centering
    \includegraphics[scale=0.5]{Projektrapport/hierarki.png}
    \caption {Överblick av hierarki i larmsystemet. Varje basenhet innehåller en startup fil med enhetens huvudlogik. Övriga filer innehåller funktioner som bidrar till basenheterna}
    \label{fig:drawing}
\end{figure}


\section{Resultat}
I detta avsnitt redovisas resultatet för delsystemen som baseras på de mål som sattes i projektets början.

Målet med projektet var att bygga ett centraliserat larmsystem.
\subsection{Centralenhet}
Centralenheten kan konfigurera och kalibrera samtliga kopplade och på nytt anslutna periferienheter.
Utöver detta kan centralenheten kontrollera status samt antigen slå på eller av larmet. 
Centralenheten är självlärande, exempelvis när en enhet startar så skickar den ett "NEW ALIVE" meddelande med sin typ. 
Därefter blir den tilldelad ett ID-nummer av centralenheten. 
Centralenheten utför en kontinuerlig kontroll över befintliga periferienheter så att ingen blir bortkopplad eller hamnar utan ström. 
Detta sker med hjälp av "LIFESIGN" meddelanden som skickas från periferienheten. 
Centralenheten kan, efter att användaren har angett lösenordet, uppdatera tröskelvärdet för varje dörr- och sensorenhet samt ändra tröskelvärdet för avståndsmätaren. 
Den kan dessutom visa status och information på alla anslutna periferienheter.

\subsection{Periferienheter}
Ett centralt mål med projektet är periferienheterna som ingår i systemet. 
Dörrenheten upptäcker ifall en dörr står öppen och sensorenheten upptäcker ifall något passerar framför den inom ett visst avståndsintervall samt rörelse i form av vibrationer.

\subsubsection{Dörrenhet}
Dörrenhetens implementation har uppnåt de delmål som sattes upp i början av projektet. 
Enheten kan hantera åtta dörrar och för varje enskild dörr kan tidsintervallen för aktivering av globalt respektive lokalt larm justeras individuellt med ett CAN-meddelande \emph{SET DOOR ALARM TIME THRESHOLD}. 
Den kan larma lokalt med en röd lysdiod i fall dörren är öppen och dess standard tidsintervall överskrids.
Eter tidsintervallet för globalt larm överskrids skickas ett CAN-meddelande \emph{START ALARM} till centralenheten.
Enheten kan skicka de periodvisa \emph{LIFESIGN} meddelanden som tas emot av centralenheten.
Den kan även avlarmas med ett \emph{STOP ALARM} meddelande. 
Om en dörr står låst indikeras detta med en tänd grön lysdiod.

\subsubsection{Sensorenhet}
Avståndsmätaren ger förväntade resultat.
Genom testning och kalibrering med linjal verifieras att avståndsmätaren ger korrekt och konsekvent mätning av distans i cm.
Även om sensorn täcks så att ekot inte registrerar larmade den vilket ökar säkerheten i systemet. 
Ett delmål med sensorenheten är att den likt dörrenheten larmar både lokalt och globalt med avståndsmätaren vilket den kan göra. 
Avståndsintervallen för dessa går att justera via ett CAN-meddelande \emph{SENSOR DISTANCE THRESHOLD} från centralenheten.\\
\\
Ett annat mål med sensorenheten är att den detekterar vibrationer genom luft eller materialet vibrationssensorn fästs i. Sensorn plockar enkelt upp vibrationer i materialet den är fäst i och larmar globalt, men den är mindre känslig för vibrationer som färdas genom luften och missar ibland viktiga tecken på att den ska larma.

\subsection{Kommunikation via CAN och USART}

\subsection{Testning av system}
%simulering av defekt enhet överflöd idk

\section{Slutsats och diskussion}
I detta kapitlet konstateras de olika resultaten utifrån målen som sattes.
Dessutom diskuteras vad som hade kunnat göras bättre och potentiella förbättringar och idéer för vidarearbete av projektet.\\
\\
\subsection{Centralenhet}
Centralenheten var den mest vitala delen av systemet och det var viktigt att den fungerade som den skulle.
Eftersom den behövde ha koll på att alla enheter var anslutna hela tiden så var det viktigt för oss att livsteckensystemet fungerade och att centralenheten kunde avgöra när en enhet ställdes om eller när sladden till den enheten drogs ut.
En sårbarhet i systemet skulle vara ifall själva centralenheten skulle dras ut för då finns det plötsligt inte något som kan starta ett centralt larm. 
Därför skulle det i framtiden vara bra om man kunde jobba på integrationen med den ihopkopplade datorn så att om centralenheten tas ur bruk, planerat eller oplanerat, att datorn då märker det och kan larma till ett moln eller en typ av webb-baserad server där man kopplat sina telefoner och kan få notiser ifall hela larmet skulle stängas av.


\subsection{Periferienheter}
\noindent
Dörrenheten uppnådde de mål som vi satte på enheten. 
Dörrarna fungerade bra och inga speciella problem framträdde under de gånger som vi testade systemet som helhet. 
Vi ser inte att det finns så mycket mer att göra när det kommer till utvecklingen av dörrenheten då dörrarnas säkerhet inte behöver kunna annat än att hålla koll på vilka dörrar som öppnats och när de har varit öppna för länge och larmet ska gå.\\
\\
Eftersom att avståndsmätaren i nuläget bara jämför de värden som den får utan kontext och enskilt för varje mätning betyder det att det centrala larmet kan utlösas av att exempelvis en fluga flyger framför mätaren.
Det är ju så klart inte bra att ett hemlarm utlöses för att en fluga råkade flyga förbi mätaren.
Om systemet skulle förbättras hade det varit smart att ändra på avståndsmätarens sätt att analysera, sådant att den kanske måste få samma avstånd flera gånger i rad innan den börjar larma.\\
\\
Ytterligare ett problem med avståndsmätarens program är att intervallen inte är individuellt reglerbara utan det är det lokala tidsintervallet som korrigeras till att bli det önskade intervallet medans det globala intervallet bestäms som en faktor av det lokala. 
Då enheten startar sätts intervallen till standardvärden som är bundna till den första avståndsmätningen. 

\subsection{Fysiska och ekonomiska begränsningar}
\noindent
Tyvärr så kan mycket problem uppstå ifall man väljer att skapa systemet själv.
För det första så måste man räkna med att få spendera mer tid för att programmera de olika delsystemen samt för att koppla ihop allting, för att inte tala om att bygga ett protokoll från grunden.
I fallet med kortet MD407 kan det bli begränsande vilka utvecklingsmiljöer som fungerar att använda felfritt då det inte finns så mycket dokumentation på just det kortet jämfört med kanske lite vanligare kort såsom en Arduino eller ett Raspberry Pi.\\

\noindent
Fördelarna väger fortfarande tungt då det både är billigare att köpa in de diverse komponenterna och koppla själv än att köpa en prenumeration där installation ingår, samt ger en större valmöjlighet när det kommer till egenskaper. 
Man kan till exempel använda sig av en motor som man kopplar en avståndsmätare på för att kunna få den att snurra runt sin axel och på så sätt kunna få en kamera som kan se runt sig och få en bild av omgivningen i form av avstånd.\\


\newpage
\begin{appendix}
\includepdf[page=1, scale=0.8, pagecommand={\section{Testfall}}]{test_defect_unit.pdf}
\includepdf[page=-, scale=0.8]{ack.pdf}
\includepdf[page=-, scale=0.8]{Replay_attack.pdf}
\includepdf[page=-, scale=0.8]{rt_buffer.pdf}
\includepdf[page=-, scale=0.8]{testfall1.pdf}
\includepdf[page=-, scale=0.8]{testfall2.pdf}
\includepdf[page=-, scale=0.8]{testfall3.pdf}
\end{appendix}



\newpage

\section*{Källförteckning}
\printbibliography[heading=none]

\end{document}